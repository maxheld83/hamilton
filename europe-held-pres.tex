%preamble
	\documentclass[]{beamer}
	\usepackage{../dotfiles-etc/mystyle-pres}

%\includeonly{./tex/proposal-phd}

	%\hypersetup
	%	{pdftitle=((Mis)understanding the Fine Print of the Social Contract: The Pluralist and Deliberative Politics of Taxation),
	%	pdfauthor=(Maximilian Held), pdfcreator=(Maximilian Held), pdfsubject=(PhD Thesis),
	%	colorlinks=true, linkcolor=maxgreen, citecolor=black, filecolor=maxgreen, urlcolor=maxgreen}%change title here, too.

\title[EU]{Europe}
\subtitle[Kurzform]{A PhD Proposal}
\author[Max]{Maximilian Held}
\institute[BIGSSS]{Bremen International Graduate School of Social Sciences, Universität Bremen \& Jacobs University Bremen}
\date[26.05.06]{26. Mai 2006}
%\titlegraphic{\includegraphics[width=2cm,height=2cm]{hulogo}}
\subject{Berechenbarkeit}
\keywords{Gödelsätze, Unentscheidbarkeit}

\begin{document}

\begin{frame}
\titlepage
\end{frame}

\begin{frame}
	\frametitle{Table of Contents}
	\tableofcontents %[currentsection]
\end{frame}

%markt vs staat
%this pays for a lot of things;
%negative integration Die bereits vollendete Liberalisierung gefährdet die Errungenschaften der Mischwirtschaften, verschärft soziale Ungleichheit und schafft explosive makroökonomische Ungleichgewichte.
%Gegen die Folie der nationalen Mischwirtschaft erscheint die Europäische Union als unvollständiges, ungerechtes und prekäres Projekt regionaler Integration.
%Für eine europäische Mischwirtschaft braucht es eine europäisierte Steuerpolitik, gefolgt von Struktur-, Industrie- und Sozialpolitik. Steuern auf dem Niveau der reichen Volkswirtschaften sind allerdings für die armen Mitgliedsländer nicht zu leisten. Deshalb müssten die reichen Staaten, den armen Ländern zusätzliche Transfers zukommen lassen.

%Die EU als dysfunktionale Mischwirtschaft: "Auf einem Bein kann man nicht stehen"

%\section{There is a threefold crisis.}

%\section{Schlussfolgerung}

\begin{frame}
\frametitle{The Puzzle}

Ökonomische Schlussfolgerungen:
\begin{enumerate}
	\item Effiziente und faire wirtschaftliche Integration braucht immer eine \alert{intakte Mischökonomie mit unionsweiten Steuern}.
	\item Das Wohlstands- und Produktivitätsgefälle in der EU macht einheitliche Steuersätze einstweilen unmöglich; wir brauchen eine \alert{Transferunion}.
	\item Reales Entsparen, Kredit oder Asset-Blasen und Inflation können Krisen verstecken, verschieben und verschlimmern.
	\alert{Nichts ist gut, weil nominelle Variablen wie BSP oder Beschäftigung gut aussehen.}
	\item Kapitalmärkte mögen bessere Regulierung brauchen; \alert{Finanzkrisen sind aber letztlich Epiphänomene}.
\end{enumerate}

\begin{enumerate}
	\item Die Nutzen, Kosten und Bedingungen der wirtschaftlichen Integration müssen \alert{umfassend erklärt} werden.
	Zur wirtschaftlichen Integration und dem \alert{Abschied von imaginierten Gemeinschaften} wie dem Nationalstaat gibt es keine attraktive Alternative.
	\item Vollbeschäftigung (links, Nachfrageseite) \emph{und} BSP (rechts, Angebotseite) \alert{taugen beide nicht als \emph{hinreichende Ziele} für gute Politik}.
	\item \emph{Vielleicht} geht bei wirtschaftlicher Integration \alert{Tiefe vor Breite}.
\end{enumerate}

\end{frame}

\end{document}